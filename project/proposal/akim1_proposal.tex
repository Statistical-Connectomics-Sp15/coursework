\documentclass[12pt]{article}
\usepackage[margin=1in]{geometry}
\geometry{letterpaper}                  
\usepackage{graphicx}
\usepackage[hyphens]{url}
\usepackage{fancyhdr}
\pagestyle{fancy}
\usepackage{fixltx2e}
\usepackage{amsmath,amsfonts,amsthm,amssymb}
\usepackage{graphicx}
\usepackage{algorithm}
\usepackage{algorithmic}
\usepackage{url}
\usepackage[normalem]{ulem}
\usepackage[pdftex]{color}
\usepackage{varioref}
\usepackage{mathrsfs}
\usepackage{amsmath}
\labelformat{equation}{\textup{(#1)}}
\usepackage[sort&compress,colon,square,numbers]{natbib}


\usepackage{color}
\newcommand{\todo}[1]{{\color{red}{\it TODO: #1}}}
\newcommand{\jovo}[1]{{\color{green}{\it jovo: #1}}}
\newcommand{\will}[1]{{\color{blue}{\it will: #1}}}
\newcommand{\greg}[1]{{\color{cyan}{\it greg: #1}}}


\begin{document}

\begin{center}\Large \bf EN.580.694: Statistical Connectomics \\ Final Project Proposal \end{center}
\begin{center} akim1 $\cdot$  \today \end{center}
\bigskip

\section*{Effects of Spatial Resolution on Accurate Determination of Graph Connectivitiy}

\paragraph{Opportunity}
The ability to measure individual connectomes holds great promise in advancing
our knowledge of the brain and consciousness. Despite these promises, current
technology and state of knowledge prevents us from rescaling this measurement
process into a computational problem that can be solved within a reasonable
time with finite resources. The complexity of the problem inevitably poses a
challenge both in validation of data and establishment of a consensus dataset.
This proposal posits that one factor that may contribute to these challenges is
the spatial resolution at which the data is obtained.

\paragraph{Challenge}
There are no obvious technological or experimental challenges that have
prevented a study of the form proposed here from being pursued in the past.

\paragraph{Action}
In this study, a true neuron connectivity network will be generated using a
stochastic block model. The neurons will be assumed to be in a 2-dimensional
grid where each pixel will represent a neuron. In one limiting case, contiguous
pixels will be assigned one block, while in another limiting case, each pixel
will be assigned randomly to a block. These limiting cases will represent the
extremes of what would assumed to be observed physiologically. The former case
represents the homogeneous case where all neighboring neurons belong to the same
block while the latter represents the heterogeneous case where no two
neighboring neurons belong to the same block. Once the true network is
generated, the network will be coarsened by combining the connectivity of
neighboring neurons to form one aggregate neuron. Analysis will be performed on
this coarsened data to determine the parameters of the model and their
deviations from the true value.

\paragraph{Resolution}
It is predicted that the homogeneous case will experience minimal detriment from
coarsening while the heterogeneous case will experience significant detriment.
Since physiological conditions fall somewhere between these two limting cases,
it is expected that deviations from the true connectome will depend both on the
spatial resolution of the measurement and the brain region of interest.

\paragraph{Future Work}
Future work will include further development of the model onto a 3-dimensional
grid which would be more experimentally and clinically relevant.

\pagebreak
\subsection*{Statistical Decision Theoretic}
\begin{description}
\item[Sample Space]
$\mathscr{G} = (V,E,B)^{N\times N}$ where $V$ is the
presence of a vertex, $E$ is the presence of an edge, and $B$ is the block to
which it belongs.

\item[Model] 
The model is given by the stochastic block model:\\
$\{\text{SBM}_N^k(\rho,\beta): \rho \in\Delta_k
\text{, }\beta\in(0,1)^{k\times k}\}$

\item[Action]
Action space will consist of the possible parameters of the stochastic block
model.

\item[Decision Rule Class]
k-means clustering will be used to identify the identity of a neuron to a
particular block.

\item[Loss Function]
Likelihood or sum of squares will be used as loss function.

\item[Risk]
The risk will be defined as the expectation of the loss.
\end{description}
\end{document}  

