\documentclass{article}

\usepackage{amsmath}

\title{Statistical Connectomics HW 2}
\author{David Lee}
\date{February 17, 2015}

\begin{document}
\maketitle

A statistical decision problem we discussed in class can be divided into 6 parts - the sample space, the model, action space, decision rule class, loss function and risk function.

\begin{description}
\item[Sample space]
The sample space consists all possible arrangements of edges, nodes, and labels.
\begin{equation}
G_n=(V,E,Y)
\end{equation}

\item[Model]
A stochastic block model consisting of 
\begin{equation}
P=SBM_n^k(\rho,\beta)\textrm{ where }\rho\in\Delta_2\textrm{ and }\beta\in(0,1)^{2\times2}
\end{equation}
\begin{equation*}
\textrm{ when } k =2
\end{equation*}

\item[Action space]
The action space is the assignment given by the clustering algorithm.
\begin{equation}
A=\{y\in\{0,1\}^n\}
\end{equation}

\item[Decision rule class]
The decision rule class is given by a method of clustering (such as k-means in our case discussed in class)

\item[Loss function]
The loss function can be given by the adjusted rand index and gives the cost associated with an action
\begin{equation}
l:G_n\times A\to R_+
\end{equation}

\begin{equation*}
l=\sum_{i=1}^n\Theta(\hat{y}_i=y_i)
\end{equation*}

\item[Risk function]
The risk function is given by the following equation
\begin{equation}
R=P\times l
\end{equation}which can be redefined as the expected value of the loss function:
\begin{equation}
	R=E\{l\}
\end{equation}
\end{description}
\end{document}	