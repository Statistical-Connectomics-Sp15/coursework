\documentclass[12pt]{article}
\usepackage[margin=1in]{geometry}
\geometry{letterpaper}                  
\usepackage{graphicx}
\usepackage[hyphens]{url}
\usepackage{fancyhdr}
\pagestyle{fancy}
\usepackage{fixltx2e}
\usepackage{amsmath,amsfonts,amsthm,amssymb}
\usepackage{graphicx}
\usepackage{algorithm}
\usepackage{algorithmic}
\usepackage{url}
\usepackage[normalem]{ulem}
\usepackage[pdftex]{color}
\usepackage{varioref}
\usepackage{mathrsfs}
\usepackage{amsmath}
\labelformat{equation}{\textup{(#1)}}
\usepackage[sort&compress,colon,square,numbers]{natbib}


\usepackage{color}
\newcommand{\todo}[1]{{\color{red}{\it TODO: #1}}}
\newcommand{\jovo}[1]{{\color{green}{\it jovo: #1}}}
\newcommand{\will}[1]{{\color{blue}{\it will: #1}}}
\newcommand{\greg}[1]{{\color{cyan}{\it greg: #1}}}


\begin{document}

\begin{center}\Large \bf EN.580.694: Statistical Connectomics \\ Final Project Proposal \end{center}
\begin{center} Greg Kiar $\cdot$  \today \end{center}
\bigskip

\section*{Searching for a brain parcellation atlas which maximizes test-retest reliability of same-subject MR connectomes}

\paragraph{Opportunity}
As a current state of the art, human connectomes can be estimated from Diffusion Weighted MR imaging (DWI, dMRI, DTI) and structural MR imaging (sMRI, MPRAGE). When estimating these connectomes, an argument can be made that voxel-level artifacts and imperfect registration procedures cause sufficient noise that full scale graphs are unreliable for node-wise analysis.  For this reason, so-called small graphs can be produced which combine nodes in the full graphs, voxels in the structural image, into defined regions. These region sets are termed atlases.
\paragraph{Challenge}
The atlases used are defined often with knowledge of functional or physiological data \todo{cite}, without knowledge of structural connectivity data. As a result, it is difficult to know whether or not an atlas is an appropriate or useful parcellation of the brain for connectomics purposes. When analyzing the performance of these partitions for such connectivity data, difficult graph statistics must be employed.
\paragraph{Action}
Test-retest (TRT) reliability is a measure which seeks to compare connectomes estimated for the same subject across different scans to the remainder of the dataset. A successful TRT test results in the same subject graphs being more similar to one another than all other graphs. Here, we propose and test several methods TRT in order to evaluate the performance of three commonly used atlases; Desikan \todo{cite}; Destrieux \todo{cite}; DKT \todo{cite}.
\paragraph{Resolution}
We will gain insight into which of the tested atlases best partitions estimated connectomes, and thus which atlas should be used to assess the quality of estimation and for basic classification tasks. We will also see which TRT metric is best suited for comparing the resulting connectomes.
\paragraph{Future Work}
One likely difficulty of this project will be developing TRT metrics that are not biased heavily by sparse data (few edges within the graph). As we seek to answer two questions, it may be possible to gain more insightful solutions by attempting to answer each question independently in future.

\pagebreak
\subsection*{Statistical Decision Theoretic}
Each subject has a value $s$, where $m$ is the total number of subjects, meaning that $2 \times m$ is the total number of graphs. The large graph for each subject, $G_s$, with nodes $V_s$ and edges $E_s$ is parcellated by an atlas, $k_i$, and produces the resulting small graph with labels $Y_S$. The decision rule class, for which the several methods have yet to be set, will determine the similarity between graphs. The loss is determined by incorrect matching of a graph to the same subject, and the risk is the expected loss over the space of all graphs.


\begin{description}

\item[Sample Space] $(A, P) \in \mathcal{A}_N \times \mathcal{P}_N = \mathscr{G}_N $, where 
 Adjacency Matrices: $A = \{0,1\}^{N \times N}$
and
Vertex Positions: $P= \{P_i\}$, where $P_i = (x_i,y_i,z_i) \in \mathbb{R}^3$.

\item[Model]
$  
\{SBM^{k_i}_N (\rho, \beta):$ 
$\rho \in \Delta_{k_i}$ 
$\beta \in (0,1)^{k_i \times k_i} \}$

\item[Action Space]

$ \mathcal{W}_{n} \times \mathcal{W}_{n}$, where 
$W \in \mathcal{Z}^{n \times n}$, 
$\forall$ n corresponding to three different atlases,
$\{Desikan, {Destrieux}, {DKT}\}$, and $\mathcal{Z} = \{0,1,2,...\}$.  In words, $W$ is a weighted adjacency matrix.

\item[Decision Rule Class]
$ f: \mathcal{A}_N \to \mathcal{W}_n$

\item[Loss Function]
$ \ell(\hat{W}^k_i, \hat{W}^k_j) = \lvert \lvert \hat{W}^k_i - \hat{W}^k_j \rvert \rvert^2_F$



\item[Risk Function]
$ \\ R = E[L] = \sum\limits_{j}^{\mathscr{G}} L(G, G_j) \cdot p(G)$ $/$ $\lvert \mathscr{G} \rvert$

\end{description}

\end{document}  

